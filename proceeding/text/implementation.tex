\section{Implementation}

Scientific Python packages Astropy \cite{astropy} and Gammapy were used to generate all of the catalog and source data on \gammasky. This step includes preparing formatted JSON files from fetched raw FITS data for display in our Catalog View panels. Serving our catalog data in such manner requires us to frequently update the ``live" catalogs SNRcat and gamma-cat stored in our server. All data is consumed with the JavaScript and HTML front-end. The website's architecture was organized using Angular 2\footnote[5]{\url{https://angular.io/}}, a modern web application framework for JavaScript. Using Angular 2 has allowed us to compile the site into a single-page application. The sphere interface and visualization in the Map View page was implemented under the Aladin Lite tool \cite{aladin-lite} developed at CDS. The website is being hosted by GitHub Pages.


\gammasky exists as a static web page, meaning that there is no back-end server doing real-time data processing or content generating on request. The JavaScript client is still capable of downloading any static assets, including very large HiPS survey images and catalog datasets. Advantages to the static website technology include easy implementation and maintenance, as well as efficient data loading within the GitHub Pages server.
