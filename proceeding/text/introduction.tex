\section{Introduction}

The field of gamma-ray astronomy is growing tremendously – while only a decade ago no more than a handful of sources were observed in the GeV range, today thousands have been found, including hundreds within the TeV range. This advancement has been made possible due to novel ground-based Cherenkov telescope instruments for very-high-energy (VHE; $>$ 100 GeV) source detections. Such systems exhibit more accurate detections and higher angular resolutions than ever before. Space-based satellites sharing similar technological breakthroughs have further developed the high-energy (HE; 10 MeV -- 100 GeV) range of gamma-ray astronomy, as can be observed in the latest images from the Fermi Large Area Telescope (Fermi-LAT). As a whole, the instruments can capture gamma-rays in a wide spectrum of energies from 10 MeV to 10 TeV. The High Energy Stereoscopic System (H.E.S.S.) Galactic Plane Survey \cite{hgps}, the High-Altitude Water Cherenkov Observatory (HAWC) 1st Year Catalog, and the fourth Fermi-LAT Point Source Catalog (4FGL) are among the highly anticipated surveys that will be unveiled in the near future. Furthermore, with an incoming wave of notable systems planned to operate soon, such as the ground-based Cherenkov Telescope Array (CTA) for TeV observations \cite{cta}, numerous never-before-seen sources in the gamma-ray sky are expected to be discovered. With such abundance of HE and VHE sources and a rapid growth of interest in this field, there is an increasingly evident need for a central hub for exploring relevant catalog and image data across the gamma-ray sky. Our website (\url{http://gamma-sky.net}) was designed to function as such.
