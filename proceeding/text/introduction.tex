\section{Introduction}

\begin{itemize}

\item Evolution of VHE gamma-ray astronomy - increasing number of detections, novel Cherenkov telescope arrays (especially CTA)

\item Upcoming surveys like HGPS (by MPIK) - clearer resolution than our current surveys

\item Because of an increasing interest in the field, there is a need for a hub of VHE data (GeV, TeV) across many different catalogs.
This is what gamma-sky.net was created for.

\end{itemize}


    The field of very-high-energy (VHE) astronomy is growing tremendously – while only
    a decade ago we no more than a handful of sources in the GeV range, today we have thousands, including
    hundreds within the TeV range. This advancement has been made possible due to our novel ground-based Cherenkov telescope instruments. Such systems have a higher angular resolution than ever before. Space-based satellites for missions in high-energy (HE) astronomy also share similar advancements, as can be observed in the latest images from the Fermi Large Area Telescope (Fermi-LAT). As a whole, these instruments can capture gamma-rays in a wide spectrum of energies from 10 MeV to 10 TeV. Many notable instruments are expected to unveil new surveys in the near future. Such surveys include the
    High Energy Stereoscopic System (H.E.S.S.) Galactic Plane Survey and the High-Altitude Water Cherenkov Observatory (HAWC) 1st Year Catalog.
    Furthermore, with an incoming wave of systems planned to operate soon, such as the ground-based Cherenkov Telescope Array (CTA),
    we will expect to discover numerous never-before-seen sources in the gamma-ray sky. With such abundance of VHE sources and
    rapid growth of interest in gamma-ray astronomy, there is an evident need for a central hub for all relevant catalog and image data.
    Our website (\url{http://gamma-sky.net}) was designed to function as such.
