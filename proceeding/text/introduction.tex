\section{Introduction}

% \begin{itemize}
%
% \item Evolution of VHE gamma-ray astronomy - increasing number of detections, novel Cherenkov telescope arrays (especially CTA)
%
% \item Upcoming surveys like HGPS (by MPIK) - clearer resolution than our current surveys
%
% \item Because of an increasing interest in the field, there is a need for a hub of VHE data (GeV, TeV) across many different catalogs.
% This is what gamma-sky.net was created for.
%
% \end{itemize}


The field of very-high-energy (VHE) astronomy is growing tremendously – while only a decade ago we observed no more than a handful of sources in the GeV range, today we have thousands, including hundreds within the TeV range. This advancement has been made possible due to our novel ground-based Cherenkov telescope instruments. Such systems exhibit more accurate source detections higher angular resolutions than ever before. Space-based satellites sharing similar technological breakthroughs have further developed the high-energy (HE) range of gamma-ray astronomy, as can be observed in the latest images from the Fermi Large Area Telescope (Fermi-LAT). As a whole, the instruments can capture gamma-rays in a wide spectrum of energies from 10 MeV to 10 TeV. The High Energy Stereoscopic System (H.E.S.S.) Galactic Plane Survey \cite{hgps}, the High-Altitude Water Cherenkov Observatory (HAWC) 1st Year Catalog, and the fourth Fermi-LAT Point Source Catalog (4FGL) are among the highly anticipated surveys that will be unveiled in the near future. Furthermore, with an incoming wave of notable systems planned to operate soon, such as the ground-based Cherenkov Telescope Array (CTA), we expect to discover numerous never-before-seen sources in the gamma-ray sky. With such abundance of HE and VHE sources and a rapid growth of interest in gamma-ray astronomy, there is an evident need for a central hub of all relevant catalog and image data. Our website (\url{http://gamma-sky.net}) was designed to function as such.

TODO: Add citations for: HAWC first year catalog, 4FGL, CTA? Or should we remove these references? Maybe for some we can reference just websites like TeVCat did in their paper?
