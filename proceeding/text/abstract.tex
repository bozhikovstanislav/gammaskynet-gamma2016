\url{http://gamma-sky.net} is a novel interactive website designed for exploring the gamma-ray sky. The Map View portion of the site is powered by the Aladin Lite sky atlas, providing a scalable survey image tesselated onto a three-dimensional sphere. The map allows for interactive pan and zoom navigation as well as search queries by sky position or object name. The default image overlay shows the gamma-ray sky observed by the Fermi-LAT gamma-ray space telescope. Other survey images (e.g. Planck microwave images in low/high frequency bands, ROSAT X-ray image) are available for comparison with the gamma-ray data.
Sources from major gamma-ray source catalogs of interest (Fermi-LAT 2FHL, 3FGL and a TeV source catalog) are overlaid over the sky map as markers. Clicking on a given source shows basic information in a popup, and detailed pages for every source are available via the Catalog View component of the website, including information such as source classification, spectrum and light-curve plots, and literature references. with details such as source type, literature references, spectra and light curves.

We intend for \gammasky to be applicable for both professional astronomers as well as the general public. The website started in early June 2016 and is being developed as an open-source, open data project on GitHub (\gammaskygh). We plan to extend it to display more gamma-ray and multi-wavelength data. Feedback and contributions are very welcome!
