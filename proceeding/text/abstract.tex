\url{http://gamma-sky.net/} is a novel interactive website, created specifically for exploring the gamma-ray sky. The map view is powered by the Aladin Lite sky atlas, which allows interactive pan and zoom navigation as well as search by sky position or object names, similar to Google maps. The initial image shows the gamma-ray sky observed by the Fermi-LAT gamma-ray space telescope, other survey images (e.g. Planck radio image or ROSAT X-ray image) are available for comparison with the gamma-ray data.
Sources from major gamma-ray source catalogs of interest (Fermi-LAT 2FHL, 3FGL as well as a TeV source catalogs) are overlaid as markers. Clicking on a given source shows some information in a popup, and detailed pages for every source are available with details such as source type, literature references, spectra and light-curves.

We want gamma-sky.net to be useful for both professional astronomers as well as the general public. The website was started only very recently. It is being developed as an open-source, open data project on Github (\url{https://github.com/gammapy/gamma-sky}). We plan to extend it to display more gamma-ray and multi-wavelength data. Feedback and contributions are very welcome!
