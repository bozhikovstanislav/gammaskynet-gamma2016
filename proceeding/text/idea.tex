\section{Idea}

\gammasky is a one-stop resource for browsing images and catalogs but also for closely examining a specific gamma-ray source. Although it was mainly built for the greater astronomical community, the webpage additionally targets the general public through a user-friendly interface and a clean information layout, all of which are compiled under cutting-edge web tools.

\begin{figure}[tb]
\centerline{\includegraphics[width=\textwidth]{figures/mapview_wide}}
\caption{Map View of \gammasky showing the gamma-ray sky observed by the Fermi-LAT space telescope, broken into three bands: 0.3 - 1 GeV (red-band), 1 - 3 GeV (green-band), and 3-300 GeV (blue-band). The pop-up for one TeV source is displayed, as well as the toolbar on the side containing widgets for controlling the map overlays.}
\label{fig:mapview}
\end{figure}

\begin{figure}[tb]
\centerline{\includegraphics[width=\textwidth]{figures/catview_wide_zoom}}
\caption{Catalog View of \gammasky showing the catalog and source selection field at the top, and the source detail view for one source from the Fermi-LAT 3FGL source catalog.}
\label{fig:catview}
\end{figure}

Individuals who access \url{http://gamma-sky.net} via any modern internet browser will be welcomed with the Map View homepage. This page presents an overlay of multi-wavelength survey images, most of which are all-sky images, wrapped around a three-dimensional sphere. The map features pan-and-zoom functionality for easily navigating and quickly browsing the sky. Gamma-ray sources from our catalog data have been pinpointed onto the sphere, as shown in Figure~\ref{fig:mapview}. See Table~\ref{tab:images} for our selection of images on the website and Table~\ref{tab:catalogs} for the catalogs we display. The Map View page also utilizes a powerful search tool to either pan the view to a given sky position or locate a source by name. This functionality allows the user to easily find their sources and study their visual context in relation to other objects.  \gammasky additionally embodies a Catalog View, which incorporates more detailed information for each of the sources in our catalogs. Professional astronomers will navigate to this component of the website for the deep investigation of a particular source. See the Catalog View page in Figure~\ref{fig:catview}.

%table

\begin{table}[h]

\caption{Image information.}
\label{tab:a}
\tabcolsep7pt\begin{tabular}{ || lrlll ||}
\hline

\textbf{Image} & \textbf{Resolution (arcsec)} & \textbf{Type} & \textbf{Color?} & \textbf{Description}\\ \hline
\textbf{Fermi color} & 51.53 & gamma-ray & color & Fermi-LAT\\
\textbf{AKARI 90um} & 51.53 & infrared & grayscale & AKARI\\
\textbf{CGPS-VGPS CONT} & 51.53 & radio & grayscale & \\
\textbf{Spitzer GLIMPSE360} & 1.2 & infrared & color & Spitzer \\
\textbf{Haslam 408} & 51.53 & radio & grayscale & 408 MHz\\
\textbf{IRIS Band 4-100um} & 51.53 & infrared & grayscale & IRIS \\
\textbf{Planck R2 LFI Color 30-44-70 GHz} & 51.53 & microwave & color & Planck 30-44-70 GHz\\
\textbf{Planck R1 + R2 HFI Color 353-545-857 GHz} & 51.53 & microwave & color & Planck 353-545-857 GHz\\
\hline
\end{tabular}

\end{table}


% Table
\begin{table}[tb]

\caption{
Source catalogs currently displayed on \gammasky .
We plan to add other catalogs of interest to gamma-ray astronomers in the future,
e.g. the upcoming H.E.S.S. and HAWC TeV source catalogs, or the ATNF pulsar catalog.
}
\label{tab:catalogs}
\tabcolsep7pt\begin{tabular}{ lrll }
\hline
Catalog   & Sources & Updates    & Description \\
\hline
gamma-cat &     153 & continuous & Open TeV gamma-ray source catalog  \\
&&& \gammacat  \\
2FHL      &     360 & fixed      & Second Fermi-LAT catalog of high-energy sources \citep{2fhl}\\
&&& \url{http://fermi.gsfc.nasa.gov/ssc/data/access/lat/2FHL/}  \\
3FGL      &    3034 & fixed      & Third Fermi-LAT point source catalog \citep{3fgl}\\
&&& \url{http://fermi.gsfc.nasa.gov/ssc/data/access/lat/4yr_catalog/}  \\
SNRcat    &     378 & continuous & A census of high-energy observations of Galactic supernova remnants \citep{snrcat}\\
&&& \url{http://www.physics.umanitoba.ca/snr/SNRcat/} \\
\hline
\end{tabular}
\end{table}


\gammasky aims to be a resource for analyzing a source or region; however, any tools for further analysis will not be incorporated into the website. In order to continue an investigation from our site, we point users to Gammapy \cite{gammapy}, a Python package for gamma-ray astronomy, which can used to run scripts locally. These scripts will generate the detailed plots and noteworthy results that astronomers are interested in. Alternatively, users may navigate to TeVCat \cite{tevcat} for investigating TeV sources, or the ASI Science Data Center (ASDC) website for their online tools (e.g. ASDC Data Explorer\footnote[1]{http://www.asdc.asi.it/tutorial/DataExplorer/DataExplorerTutorial.html}, SED Builder\footnote[2]{http://www.asdc.asi.it/tutorial/SEDBuilder/SEDBuilderTutorial.html}) and their TeGeV Catalogue for TeV sources \cite{tgevcat}.


It is imperative for all of our data on the online portal to be openly available for download and local analysis by any user. Additionally, \gammasky is an entirely open-source project and other developers are welcome to contribute to the code. We advise those interested in contributing to visit our GitHub repository at \url{https://github.com/gammapy/gamma-sky}.
