\section{Idea}

\begin{itemize}

\item Interactive website designed for exploring the gamma-ray sky

\item Survey images of different frequency bands (mainly all-sky) overlaid onto a three-dimensional map. Gamma-ray sources from catalog data are pinpointed onto the map.

\item The website facilitates both quick browsing and deep investigation of sources

\item Understand the context of sources by viewing them on the map

\item Easily compare sources from different catalogs

\item Website targets professional astronomers, but also the general public through
a user-friendly interface and the easily understandable layout of sources plotted over a map.

\item Open source, open data - allows for 1. users to download any data from our website, and 2. for other developers to contribute to the code.

\end{itemize}


    gamma-sky.net is a one-stop resource for perusing images and catalogs but also closely examining a specific gamma-ray source.
    Although it was mainly built for the astronomical community, the webpage also targets the general public through a
    user-friendly interface and a clean information layout, all of which is compiled under cutting-edge web tools.

    Individuals who access the website via any modern internet browser will be welcomed with the Map View page.
    We overlaid multi-wavelength survey images, most of which are all-sky images, onto a three-dimensional sphere.
    The map features pan-and-zoom functionality for easily navigating and quickly browsing the sky. Gamma-ray sources from
    our catalog data have been pinpointed onto the sphere, as shown in Figure~\ref{fig:mapview}. This setup allows the user to understand
    the visual context of sources in relation to one another. It also enables the user to easily contrast sources across catalogs.
    gamma-sky.net additionally embodies a Catalog View, which incorporates more detailed information for each of the sources in our catalogs.
    Professional astronomers will navigate to this component for the deep investigation of a particular source.
    See Figure~\ref{fig:catview} for the catalog view.

    It is imperative for our online portal and data to be open-source; all of the catalog and image data on gamma-sky.net are
    openly available for download, and other developers are encouraged to contribute to the code. We advise those interested
    in contributing to visit our Github repository at \url{https://github.com/gammapy/gamma-sky}.
