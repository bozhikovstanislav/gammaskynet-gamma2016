\section{Idea}

% \begin{itemize}
%
% \item Interactive website designed for exploring the gamma-ray sky
%
% \item Survey images of different frequency bands (mainly all-sky) overlaid onto a three-dimensional map. Gamma-ray sources from catalog data are pinpointed onto the map.
%
% \item The website facilitates both quick browsing and deep investigation of sources
%
% \item Understand the context of sources by viewing them on the map
%
% \item Easily compare sources from different catalogs
%
% \item Website targets professional astronomers, but also the general public through
% a user-friendly interface and the easily understandable layout of sources plotted over a map.
%
% \item Open source, open data - allows for 1. users to download any data from our website, and 2. for other developers to contribute to the code.
%
% \end{itemize}


    gamma-sky.net is a one-stop resource for browsing images and catalogs but also for closely examining a specific gamma-ray source.
    Although it was mainly built for the greater astronomical community, the webpage additionally targets the general public through a
    user-friendly interface and a clean information layout, all of which are compiled under cutting-edge web tools.

    Individuals who access the website via any modern internet browser will be welcomed with the Map View page.
    This page presents an overlay of multi-wavelength survey images, most of which are all-sky images, wrapped around a three-dimensional sphere.
    The map features pan-and-zoom functionality for easily navigating and quickly browsing the sky. Gamma-ray sources from
    our catalog data have been pinpointed onto the sphere, as shown in Figure~\ref{fig:mapview}. The Map View page also utilizes a powerful search tool to either pan the view to a given sky position or locate a source by name. This functionality allows the user to easily find their sources and study their visual context in relation to other objects.
    gamma-sky.net additionally embodies a Catalog View, which incorporates more detailed information for each of the sources in our catalogs.
    Professional astronomers will navigate to this component of the website for the deep investigation of a particular source.
    See the Catalog View page in Figure~\ref{fig:catview}.

    It is imperative for all of our data on the online portal to be openly available for download and local analysis by any user. Additionally, gamma-sky.net is an entirely open-source project and other developers are welcome to contribute to the code. We advise those interested in contributing to visit our GitHub repository at \url{https://github.com/gammapy/gamma-sky}.
