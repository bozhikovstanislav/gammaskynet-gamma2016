\renewcommand{\thefootnote}{\fnsymbol{footnote}}

\section{Features}

The items listed below illustrate the interface of \gammasky and its utilization as a tool for astronomers.

\begin{enumerate}

\item \textbf{Browsing and navigation features} in the Map View component:

\begin{itemize}
\item Pan and zoom
\item Search tools - locate objects by name, association, or coordinate position
\item Toggle and view specific catalog layers and sky images
\item Pop-up information over each source
\item Export and share images from the sky map (in PNG format)
\end{itemize}

\item \textbf{Analysis tools} in the Catalog View component\footnote[1]{Some of the features listed for the Catalog View are currently only available for select source catalogs, but they are expected to be a part of all catalogs in the near future.}:

\begin{itemize}

\item Search and select a source by its name
\item Basic information - position, association and class
\item Extension information
\item Spectral index, brightness and flux
\item Distance and redshift
\item Graphs of light curves and emission spectra
\item Detection/observation information - instrument, date of discovery and literature references

\end{itemize}

\end{enumerate}
