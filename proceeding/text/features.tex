\renewcommand{\thefootnote}{\fnsymbol{footnote}}

\section{Features}

The items listed below illustrate the interface of gamma-sky.net and its utilization as a tool for astronomers.

\begin{enumerate}

\item \textbf{Browsing and navigation features} in the Map View component:

\begin{itemize}
\item Pan and zoom
\item Search tools - locate objects by name, association, or coordinate position
\item Toggle and view specific catalog layers and sky images
\item Pop-up information over each source
\item Export and share images from the sky map (in PNG format)
\end{itemize}

\item \textbf{Analysis tools} in the Catalog View component\footnote[1]{Some of the features listed for the Catalog View are currently only available for select catalogs, but they are expected to be a part of all catalogs in the near future.}:

\begin{itemize}

\item Search and select a source by its name
\item Basic information - position, association and class
\item Extension information
\item Spectral index, brightness and flux
\item Distance and redshift
\item Graphs of light curves and emission spectra
\item Detection/observation information - instrument, date of discovery and relevant papers

\end{itemize}

\end{enumerate}

TODO: How does this features list look? Should we make it more ``elegant"/space them out a bit?

\begin{figure}[t]
\centerline{\includegraphics[width=\textwidth]{figures/mapview_wide}}
\caption{Map View.}
\label{fig:mapview}
\end{figure}

\begin{figure}[t]
\centerline{\includegraphics[width=\textwidth]{figures/catview_wide_zoom}}
\caption{Catalog View.}
\label{fig:catview}
\end{figure}
