\renewcommand{\thefootnote}{\fnsymbol{footnote}}


\section{Features}

(Should we present this information in a bullet list or in paragraphs?)

\begin{enumerate}

\item Map View - easy navigation and quick browsing

  \begin{itemize}

  \item Pan and zoom the sky map by dragging and scrolling

  \item Toggle and view specific catalog layers and multi-wavelength survey images

  \item Pop-ups over each source for basic information

  \item Powerful search tools - locate objects by name, association, or coordinate position

  \item Export and share a specific view of the sky map via PNG

  \end{itemize}


\item Catalog View - deeply investigate a specific source

  \begin{itemize}

  \item Search tool to find a source in its respective catalog by source name

  \item Basic info, extension info, spectral info, distance info

  \item Light curves, emission spectra (currently only for 3FGL catalog)

  \item References to which telescope detected the source and links to where our data came from

  \end{itemize}


\item Further analysis of our data with tools like Gammapy (generate specific plots, etc.)

\item Mention again that all data is openly available for download

\end{enumerate}

\begin{figure}[tb]
  \centerline{\includegraphics[width=\textwidth]{figures/mapview_wide}}
  \caption{Map View.}
  \label{fig:mapview}
\end{figure}

\begin{figure}[tb]
  \centerline{\includegraphics[width=\textwidth]{figures/catview_wide_zoom}}
  \caption{Catalog View.}
  \label{fig:catview}
\end{figure}


  The items listed below illustrate the interface and utilization of gamma-sky.net as a tool for astronomers.

  \begin{enumerate}

    \item \textbf{Browsing and navigation features} as a part of the Map View component of the website:

      \begin{itemize}

        \item Pan and zoom
        \item Search tools - locate objects by name, association, or coordinate position. These user inputs are interpreted by the Sesame service, a search term resolver for astronomical objects which queries several databases and returns the resolved sources. Both Sesame and the databases searched (Simbad, NED, and VizieR) are maintained by the Centre de Donn\'{e}es astronomiques de Strasbourg (CDS).
        \item Toggle and view specific catalog layers and sky images
        \item Pop-up information over each source
        \item Export and share images from the sky map (in PNG format)

      \end{itemize}

    \item \textbf{Analysis tools} under gamma-sky.net's Catalog View\footnote[1]{Some of the features listed for the Catalog View are currently only available for select catalogs, but they are expected to be a part of all catalogs in the near future.}:

      \begin{itemize}

        \item Search and select a source by its name
        \item Basic information - position, association and class
        \item Extension information
        \item Spectral index, brightness and flux
        \item Distance and redshift
        \item Graphs of light curves and emission spectra
        \item Detection/observation information - instrument, date of discovery and relevant papers

      \end{itemize}

  \end{enumerate}
